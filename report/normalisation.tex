\subsection*{ex7}
\subsubsection*{Required for first normal form}
\begin{itemize}
    \item Elements are atomic
    \item No repeating groups
\end{itemize}
\subsubsection*{Making the data first normal form}
The faculties relation is already in first normal form because every relation is atomic (each element is only one piece of data) and there are no repeating groups (each column is unique). We can keep our relation:\\
\\FACULTIES(\underline{faculty},\underline{building},\underline{room},capacity,\underline{lecturer\_email},lecturer\_firstname,lecturer\_surname)\\
\\However, the students relation is not first normal form. Firstly, the address column is not atomic because it contains the postcode and street name. Secondly, there are repeated groups. The lecturer columns are repeated (lecturer1, lecturer2) and the coursework columns are repeated (coursework1, coursework2, coursework3). Therefore, we need a new set of relations:\\
\\STUDENTS(student firstname,student surname,\underline{student id},student email,year,street,postcode,contact number)\\
MODULES(\underline{module id}, module name, leader)\\
STUDENTMODULES(\underline{student id}, \underline{module id}, exam mark)\\
LECTURERMODULES(\underline{lecturer email}, \underline{module id})\\
COURSEWORKMARKS(\underline{coursework id}, student id, module id, mark)\\


\subsection*{ex8}
\subsubsection*{Partial-key dependencies}
A partial key is when one of the non-key columns depends on only a part of a composite key. There are partial-key dependencies in our data:
\begin{itemize}
    \item $building, room \rightarrow capacity$
    \item $lecturer\_email \rightarrow lecturer\_firstname, lecturer\_surname$
    \item $building \rightarrow faculty$
    \item $lecturer\_email \rightarrow faculty$
\end{itemize}

\subsubsection*{Required for second normal form}
\begin{itemize}
    \item No partial key dependencies (as described above)
\end{itemize}

\subsubsection*{Our data in second normal form}
FACULTIES(\underline{faculty})\\
BUILDINGS(\underline{building}, \underline{faculty})\\
ROOMS(\underline{building}, \underline{room}, capacity)\\
LECTURERS(\underline{lecturer\_email},lecturer\_firstname,lecturer\_surname)\\
LECTURERFACULTY(\underline{lecturer\_email}, \underline{faculty})\\
STUDENTS(student firstname,student surname,\underline{student id},student email,year,street,postcode,contact number)\\
MODULES(\underline{module id}, module name, leader)\\
STUDENTMODULES(\underline{student id}, \underline{module id}, exam mark)\\
LECTURERMODULES(\underline{lecturer email}, \underline{module id})\\
COURSEWORKMARKS(\underline{coursework id}, student id, module id, mark)\\



\subsection*{ex9}
\subsubsection*{Transitive dependencies}
A transitive dependency is when a non-key attribute depends on another non-key attribute, which in turn depends on the primary key.

\subsubsection*{Required for third normal form}
\begin{itemize}
    \item For every non-trivial functional dependency, $A \rightarrow B$, A is a superkey or B is a prime attribute
    \item Attributes are determined only by the keys
    \item No transitive dependencies
\end{itemize}
\subsubsection*{Our data in third normal form}
As it stands, our data is in third normal form. \textbf{Before} we normalised to second normal form, we had the following transitive dependency:\\
\\$(building, room) \rightarrow lecturer\_email \rightarrow faculty$\\
\\However, this transitive dependency was dealt with when we converted our data to second normal form. Therefore, our relations remain the same:\\
\\FACULTIES(\underline{faculty})\\
BUILDINGS(\underline{building}, \underline{faculty})\\
ROOMS(\underline{building}, \underline{room}, capacity)\\
LECTURERS(\underline{lecturer\_email},lecturer\_firstname,lecturer\_surname)\\
LECTURERFACULTY(\underline{lecturer\_email}, \underline{faculty})\\
STUDENTS(student firstname,student surname,\underline{student id},student email,year,street,postcode,contact number)\\
MODULES(\underline{module id}, module name, leader)\\
STUDENTMODULES(\underline{student id}, \underline{module id}, exam mark)\\
LECTURERMODULES(\underline{lecturer email}, \underline{module id})\\
COURSEWORKMARKS(\underline{coursework id}, student id, module id, mark)\\